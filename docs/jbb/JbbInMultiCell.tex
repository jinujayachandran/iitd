% !TeX spellcheck = en_US 
\documentclass[10pt, a4paper, twoside,fleqn]{article}
\usepackage[left=2cm, right=2cm, top=1.8cm, bottom=3cm]{geometry}
\usepackage[utf8]{inputenc}
\usepackage[english]{babel}
\usepackage{cite}
\usepackage[title]{appendix}
\usepackage{graphicx}
\usepackage{color}
\usepackage{amsmath}
\usepackage{amssymb}
\usepackage{amsfonts}
\usepackage{relsize}
\usepackage{graphicx}  
\usepackage{subcaption}
\usepackage{url}
\usepackage{float}
\usepackage{mathtools}
\usepackage{bm}
% When building with pdftex this will hyperref with dvipdfm option will not work. 
% Uncomment the statement if building with latex->dvi2pdf
%\usepackage[dvipdfm]{hyperref}

\graphicspath{{./fig/}}

\title{Joint Beamforming and Broadcasting in Massive MIMO for Multicell Systems}
\author{}
\date{May 2018}
\DeclareMathOperator{\Tr}{Tr}
\begin{document}
\maketitle
%\begin{titlepage}
%   \vspace*{\stretch{1.0}}
%   \begin{center}
%      \Large\textbf{Joint Beamforming and Broadcasting in Massive MIMO for Multicell systems}\\
%      \large\textit{A. Thor}
%   \end{center}
%   \vspace*{\stretch{2.0}}
%\end{titlepage}
 
\section{Introduction}
	Joint Beamforming and Broadcasting (JBB) is a novel idea proposed in \cite{bib:jbb} in which data and the system information (SI) is sent simultaneously to different terminals in a cell. The terminals to which data is beamformed are called B-terminals (BT) and other terminals are called O-terminals (OT). The transmission works on the principle that if the SI is transmitted in the nullspace of channel estimate of BT then the interference to BT is negligible. Only the single cell case is dealt in \cite{bib:jbb}. In this document we extend it to multicell scenario.
%\footnote{The document is mainly for review purpose. The person reading this document is assumed to have a good knowledge on Joint Beamforming and Broadcasting or have read the paper \cite{bib:jbb} along with understanding on massive MIMO multi cell systems.}

	
	The aim of the document is to find the achievable rates of transmission for BT along with its power allocation scheme. For the OT we try to bound the rate expression. In the document the following shall be evaluated. For the BT, under JBB in multicell scenario, the channel from the base station (BS) to the terminals will be estimated. This estimation will be subject to pilot contamination. Once the channel is estimated the performance of the BT, under JBB is evaluated by considering MR precoding for the BT. The signal power and the noise/interference power is evaluated to come up with the achievable rate. Using the rate expression, a multicell power control algorithm proposed in \cite{bib:MassiveMimoBook} is used for power allocation to BT.
     For OT, the channel estimation is done during the pilot phase. In the payload phase the signal received at an OT shall be evaluated to get the signal power and noise/interference variances. An achievable rate for an OT is then evaluated.

\section{Approach and Assumptions}	
	In any particular cell the data is sent to BT and OT simultaneously. Also all the BSs transmit data to their respective terminals at the same time and on the same frequency (i.e, frequency reuse factor is one). The BS has imperfect channel state information (CSI) of the BTs but not for OTs. An OT learns its own channel with the help of pilots transmitted by BS. During the pilot transmission phase we assume that there is no beamforming performed to the BTs.

	A multicell system with $L$ cells each cell having $K$ BTs which uses the same frequency band are assumed. Each BS is assumed to have $M$ antennas. The transmission from the BS and from the terminals are synchronous both in pilot phase and payload phase in all the cells. The $l$-th BS transmits the $M$-vector
\begin{eqnarray}\label{eqn:txbt}
	\pmb{x}_l(t) = \sqrt{\rho_b}\sum\limits_{k'=1}^{K}\pmb{v}_{lk'}s_{lk'}(t)
    		      + \sqrt{\rho_o}\Pi^{\perp}_{{\pmb{\widehat{G}}_l}}\pmb{z}_l(t)
\end{eqnarray}
where $\pmb{v}_{lk'}\in\mathbb{C}^{M\times 1}$ and $s_{lk'}(t)$ are the beamforming vector and information symbol of the $k'$-th BT in the $l$-th cell, $\pmb{z}_l(t) \in \mathbb{C}^{M\times 1}$  is the SI vector intended for the OT. $\pmb{G}_l \in {\mathbb C}^{K \times M}$ denotes the downlink channel from M BS antennas to K BTs in the $l$-th cell. Further, $\pmb{\widehat{G}}_l$ denotes the corresponding estimate and $\Pi^{\perp}_{{\pmb{\widehat{G}}_l}} \triangleq \pmb{I}-\pmb{\widehat{G}}_l(\pmb{\widehat{G}}_l^H\pmb{\widehat{G}}_l)^{-1}\pmb{\widehat{G}}_l^H$, is the projection matrix for the space orthogonal to the column space of $\pmb{\widehat{G}}_l$. The symbols, $s_{lk'}(t)$ are assumed to have zero mean and unit variance, and the beamforming vector $\pmb{v}_{lk'}$ depends on the estimate of the channel gain from $k'$-th BT in the $l$-th cell to the BS of the same cell. 
The power transmitted from the BS to the BTs is $\rho_b$ and therefore the beamforming vectors must satisfy
\begin{eqnarray}\label{eqn:vkcondtion}
	\mathbb{E}\left[\sum\limits_{k'=1}^{K}||\pmb{v}_{lk'}||^2\right]=1.
\end{eqnarray}
Similarly power transmitted to the OTs is $\rho_o$ and hence
\begin{eqnarray}\label{eqn:zlcondition}
	\mathbb{E}\left[||\Pi^{\perp}_{\pmb{\widehat{G}}_l}z_l(t)||^2\right]=1.
\end{eqnarray}
%The OT sees $z_l(t)$ transmitted over the channel with effective channel response $\Pi^{\perp}_{\pmb{\widehat{G}}_l}\pmb{h}_l$ which is unknown to the OT. 
We assume that $z_l(t)$ is confined to a subspace of dimension $M'$, where $M'\leq M$ i.e, $z_l(t) = \pmb{U}_l\pmb{q}_l(t)$, where $\pmb{U}_l$ is a $M\times M'$ semi-unitary matrix which is unknown to the OT. As in \cite{bib:jbb} we choose $\pmb{U}_l$ such that $\pmb{U}_l^H\pmb{U}_l=\pmb{I}$ and $\Pi_{\frac{1}{\pmb{\widehat{G}}_l}}\pmb{U}_l=\pmb{U}_l$. The vector $\pmb{q}_l(t)$ are the information symbols of the SI and are i.i.d ${\mathbb C}{\mathbb N}(0,\zeta)$ distributed. Further, the SI send from all the cells are assumed to be same i.e, $\pmb{q}_{l_1}(t) = \pmb{q}_{l_2}(t) \ \forall \ l_1,l_2 \in {1 ,\dotsc, L}$. . From \cite{bib:jbb} it also follows that $\zeta=\frac{1}{M'}$. 
\section{Performance for the B-Terminals}
In this section we derive an  expression for the downlink information rate to the BTs achieved with Maximum Ratio (MR) precoding at the BS in each cell. For MR precoding, the BS in each cell uses the channel estimate acquired during the pilot phase in uplink.  

\subsection{Channel estimation for B-Terminals} \label{sec:btchesti}
	BTs send the known pilots to the BS and BS estimates the channel. Here we are assuming that same set of orthogonal pilots are used in all the cells and all the $K$ terminals in each of the $L$ cells send their pilots simultaneously to their respective BS which results in pilot contamination.

The signal received by the $l$-th BS is given by
\begin{eqnarray}\label{eq:rcvy}
	\pmb{y}_{l} &=& \sum_{j=1}^{L} \sum_{k'=1}^{K}\sqrt{\rho_p \tau_p} \ \pmb{g}_{jk'l} \ \pmb{\phi}^H_{jk'}
        		    +  \pmb{n}_{l} \nonumber \\                    
    			   &=& \sqrt{\rho_p \tau_p} \ \pmb{g}_{lkl} \ \pmb{\phi}^H_{lk}
        			+  \sum_{\substack{k'=1 \\ k' \neq k}}^{K} \sqrt{\rho_p \tau_p} \ \pmb{g}_{lk'l} \ \pmb{\phi}^H_{lk'}
        		    +  \sum_{\substack{j=1 \\ j \neq l}}^{L} \sum_{k'=1}^{K} \sqrt{\rho_p \tau_p} \
                       \pmb{g}_{jk'l} \ \pmb{\phi}^H_{jk'}
    			    +  \pmb{n}_{l}
\end{eqnarray}
where $\pmb{g}_{jk'l} \in {\mathbb C}^{M \times 1}$ is the channel vector between the $k'$-th BT in the $j$-th cell and the $l$-th BS. The elements of $\pmb{g}_{jk'l}$ are i.i.d. ${\mathbb C}{\mathbb N}(0, \beta_{jk'l})$ distributed, where $\beta_{jk'l}$ represents the respective path loss. The pilot sequence transmitted by the $k$-th BT in the $l$-th cell is denoted by $\pmb{\phi}_{lk} \in {\mathbb C}^{\tau_p \times 1}$ where $\tau_p$ denotes the length of the pilot sequence. The sequences are orthonormal, i.e., $\pmb{\phi}_{lk}^H \pmb{\phi}_{jk'} = 0$ for  $k \ne k'$ and is $1$ for $k=k'$. The additive white gaussian noise (AWGN) vector, $\pmb{n}_l$, has i.i.d ${\mathbb C}{\mathbb N}(0,1)$ distributed elements. The transmit power in the pilot phase is represented by $\rho_p$.

For estimating the channel of the $k$-th BT of the $l$-th cell, at the BS of the $l$-th cell the received vector $\pmb{y}_{l}$ is projected onto the known pilot $\pmb{\phi}_{lk}$. Due to the orthogonality between the pilot sequences, from (\ref{eq:rcvy}) it follows that
\begin{eqnarray}\label{eqn:ylkp}
	\pmb{y}_{lk}^{p} \triangleq \pmb{y}_{l} \pmb{\phi}_{lk}
                          &=&     \sqrt{\rho_p\tau_p}\pmb{g}_{lkl}
                           +      \sum_{\substack{j=1 \\ j \neq l}}^{L} \sqrt{\rho_p\tau_p}\pmb{g}_{jkl}
             			   +      \pmb{n}_{lk}
\end{eqnarray}
where $\pmb{n}_{lk} \triangleq \pmb{n}_{l}\pmb{\phi}_{lk}$.

Let the MMSE estimate of the channel $\pmb{g}_{lkl}$ be $\pmb{\hat g}_{lkl}$. Thus $\pmb{\hat g}_{lkl} \triangleq \alpha_{lk}\pmb{y}_{lk}^{p}$. As the estimation error and the estimate are uncorrelated, $\alpha_{lk}$ satisfies
\begin{eqnarray}\label{eqn:alphammse}
	\pmb{E}[(\pmb{g}_{lkl}-\alpha_{lk}\pmb{y}_{lk}^{p})^H \pmb{y}_{lk}^{p}] = 0 \nonumber \\
	\implies \alpha_{lk} = \frac{\pmb{E}[\pmb{g}^{H}_{lkl} \pmb{y}_{lk}^{p}]}{\pmb{E}[|\pmb{y}_{lk}^{p}|^2]}
\end{eqnarray}
Substituting (\ref{eqn:ylkp}) in (\ref{eqn:alphammse}) and taking the required expectation we obtain
\begin{eqnarray}\label{eq:alphapll}
	\alpha_{lk}=\frac{\sqrt{\rho_p \tau_p} \beta_{lkl}}{1+\rho_p\tau_p\sum\limits_{j=1}^{L}\beta_{jkl}}
\end{eqnarray}
and the corresponding estimate of the channel is given by
\begin{eqnarray}\label{eq:estimatehpll}
	\pmb{\hat g}_{lkl} = \frac{\sqrt{\rho_p\tau_p}\beta_{lkl}}{1+\rho_p\tau_p\sum\limits_{j=1}^{L}\beta_{jkl}} \pmb{y}_{lk}^{p}
\end{eqnarray}
The estimation error vector is denoted by ${\pmb{\Tilde g}_{lkl}}  \triangleq  \pmb{{\hat g}}_{lkl} - \pmb{g}_{lkl}$.
The covariance matrix of the estimated channel vector is given by
\begin{eqnarray}
	\pmb{\Gamma}_{lkl} &\triangleq & \mathbb{E}[\pmb{\hat g}_{lkl}\pmb{\hat g}^H_{lkl}]
                    =       \frac{\rho_p\tau_p\beta^2_{lkl}}{1+\tau_p\rho_p\sum\limits_{j=1}^{L}\beta_{jkl}}\pmb{I} = \gamma_{lkl}\pmb{I}
\end{eqnarray}
and therefore the covariance of the estimation error is given by
\begin{eqnarray}
    \mathbb{E}[\pmb{\widetilde{g}}_{lkl} \pmb{\widetilde{g}}^H_{lkl}] &=& \beta_{lkl}\pmb{I}-\pmb{\Gamma}_{lkl} \nonumber \\
          									    &=& (\beta_{lkl} - \gamma_{lkl})\pmb{I}
\end{eqnarray}

\subsection{Achievable Information Rate for B-Terminals}
From (\ref{eqn:txbt}) it follows that the data received at the $k$-th BT of the $l$-th cell in downlink is given by
\begin{eqnarray}\label{eq:ot}
 	y_{lk}(t) = \sqrt{\rho_b}\sum_{j=1}^{L} \pmb{g}_{lkj}^{T} \left(\sum_{k'=1}^{K}\pmb{v}_{jk'}s_{jk'}(t)\right)
 		      + \sqrt{\rho_o}\sum_{j=1}^{L}\pmb{g}^T_{lkj} \Pi^{\perp}_{{\pmb{\widehat{G}}_j}} \pmb{U}_j\pmb{q}_{j}+ w_{lk}(t)
\end{eqnarray} 
where $w_{lk}(t) \sim {\mathbb C}{\mathbb N}(0,1)$  is AWGN. The beamforming vector $\pmb{v}_{jk}$ depends on the channel estimate $\pmb{\hat g}_{jkj}$.

 Considering Maximum Ratio Transmission (MR), the value of the beamforming vector for the $k$-th BT in the $j$-th cell is
\begin{eqnarray}\label{eq:bfv}
	\pmb{v}_{jk} = \sqrt{\frac{\eta_{jk}}{M\gamma_{jkj}}}\pmb{\hat g}^*_{jkj}
\end{eqnarray}
From (\ref{eqn:vkcondtion}) it follows that the power control parameters, $\eta_{jk}$ must satisfy
\begin{eqnarray}\label{eqn:etaconstraint}
\sum\limits_{k=1}^{K}\eta_{jk}=1
\end{eqnarray}
Substituting (\ref{eq:bfv}) in (\ref{eq:ot}) we get
\begin{eqnarray} \label{eqn:completeot}
	y_{lk}(t)  &=& \sqrt{\frac{\rho_b\eta_{lk}}{M\gamma_{lkl}}} \ \mathbb{E}\left[\Bigm|\Bigm| \ \pmb{\hat g}_{lkl}\Bigm|\Bigm|^2\right]s_{lk}(t)
	           +  \sqrt{\frac{\rho_b\eta_{lk}}{M\gamma_{lkl}}}\left[\Bigm|\Bigm|\pmb{\hat g}_{lkl}\Bigm|\Bigm|^2 - \mathbb{E}\left[\Bigm|\Bigm|\pmb{\hat g}_{lkl}\Bigm|\Bigm|^2\right]\right]s_{lk}(t)  \nonumber \\
	           &+& \sqrt{\rho_b} \ \pmb{\hat g}_{lkl}^T \left(\sum_{\substack{k'=1 \\ k' \neq k}}^{K} \sqrt{\frac{\eta_{lk'}}{M\gamma_{lk'l}}} \ \pmb{\hat g}^*_{lk'l} \ s_{lk'} (t)\right)
	           - \sqrt{\rho_b}\ \pmb{\widetilde g}^T_{lkl}\sum\limits_{k'=1}^{K}\sqrt{\frac{\eta_{lk'}}{M\gamma_{lk'l}}}\pmb{\hat g}^*_{lk'l}s_{lk'}(t) \nonumber \\
	           &+& \sqrt{\rho_b}\sum\limits_{\substack{j=1 \\ j\neq l}}^{L}\pmb{g}^T_{lkj}\sqrt{\frac{\eta_{jk}}{M\gamma_{jkj}}}\pmb{\hat g}^*_{jkj}s_{jk}(t)
               + \sqrt{\rho_b} \sum_{\substack{j=1 \\ j \neq l}}^{L} \pmb{g}_{lkj}^T \left(\sum_{\substack{k'=1 \\ k' \neq k}}^{K} \sqrt{\frac{\eta_{jk'}}{M\gamma_{jk'j}}} \ \pmb{\hat g}^*_{jk'j} \ s_{jk'} (t)\right)
               \nonumber \\
               &+& \sqrt{\rho_o}\sum_{j=1}^{L}\pmb{g}^T_{lkj} \Pi^{\perp}_{\pmb{\widehat{G}}_j} \pmb{U}_j \pmb{q}_{j}
               + w_{lk}(t)      
\end{eqnarray}
The first term in (\ref{eqn:completeot}) is the required signal. The second term represents uncertainty in channel gain, third term represents intra-cell interference, the fourth term stems from channel estimation errors, fifth term is the  interference from the contaminating cells called as coherent interference, sixth term represents the inter-cell interference, seventh term is the interference due to transmission to the OT and the last term is the AWGN. The signal, interference and the noise are mutually uncorrelated.

Our next aim is to find the information rate by finding the signal power and the variance of the noise/interference terms. 

The signal power (SP) can be calculated from the first term as
\begin{eqnarray}
	SP^{BT} &\triangleq&  \frac{\rho_b\eta_{lk}}{M\gamma_{lkl}}\left(\mathbb{E}\left[\Bigm|\Bigm|\pmb{\hat g}_{lkl}\Bigm|\Bigm|^2\right]\right)^2 \nonumber \\
            &=& \rho_b\eta_{lk}M\gamma_{lkl}
\end{eqnarray}
The variance of second term in (\ref{eqn:completeot}) is given by
\begin{eqnarray}\label{eq:noiseF1}
	V_1^{BT} &\triangleq& \frac{\rho_b \eta_{lk}}{M\gamma_{lkl}}   \mathbb{E}\left[\Bigm|s_{lk}(t)\left(\Bigm|\Bigm|\pmb{\hat g}_{lkl}\Bigm|\Bigm|^2- \mathbb{E}\left[\Bigm|\Bigm|\pmb{\hat  g}_{lkl}\Bigm|\Bigm|^2\right] \right)\Bigm|^2\right] \nonumber \\
	    &=& \frac{\rho_b \eta_{lk}}{M\gamma_{lkl}}\left[\mathbb{E}\left[\Bigm|\Bigm|\pmb{\hat g}_{lkl}\Bigm|\Bigm|^4\right] - \left(\mathbb{E}\left[\Bigm|\Bigm|\pmb{\hat g}_{lkl}\Bigm|\Bigm|^2\right]\right)^2 \right] \nonumber  \\
	    &=& \rho_b\eta_{lk}\gamma_{lkl}
\end{eqnarray}
In (\ref{eq:noiseF1}), since the elements of $\pmb{\hat g}_{lkl}$ have i.i.d  $CN(0,\gamma_{lkl})$ distribution, the value of $\mathbb{E}\left[\Bigm|\Bigm|\pmb{\hat g}_{lkl}\Bigm|\Bigm|^4\right]$ is evaluated to be $(M)(M+1)\gamma_{lkl}^2$.

The variance of the intra-cell interference term in (\ref{eqn:completeot})  is
\begin{eqnarray}\label{eq:noiseF2}
	V_2^{BT} &\triangleq& \rho_b \mathbb{E}\left[\left\vert \pmb{\hat g}^T_{lkl}  \left(\sum_{\substack{k'=1 \\ k' \neq k}}^{K} \sqrt{\frac{\eta_{lk'}}{M\gamma_{lk'l}}} \ \pmb{\hat g}^*_{lk'l} \ s_{lk'} (t)\right)\right\vert^2\right] \nonumber \\
        &=& \frac{\rho_b}{M} \sum_{\substack{k'=1 \\ k' \neq k}}^{K}\frac{\eta_{lk'}}{\gamma_{lk'l}}\mathbb{E}\left[ \pmb{\hat g}^T_{lk'l} \ \pmb{\hat g}^*_{lkl} \ \pmb{\hat g}^T_{lkl} \ \pmb{\hat g}^*_{lk'l} \right] \nonumber \\
	    &=&  \rho_b\gamma_{lkl}\sum\limits_{\substack{k'=1\\ k'\neq k}}^{K}\eta_{lk'}
\end{eqnarray}
The variance of channel estimation error in (\ref{eqn:completeot}) is
\begin{eqnarray}\label{eqn:noiseF4}
	V_3^{BT} &\triangleq& \rho_b \mathbb{E} \left[ \Bigm | \pmb{\widetilde{g}}_{lkl}^T \left(\sum_{k'=1}^{K} \sqrt{\frac{\eta_{lk'}}{M\gamma_{lk'l}}} \ \pmb{\hat g}^*_{lk'l} \ s_{lk'} (t)\right)\Bigm|^2 \right] \nonumber \\
     &=& \rho_b\sum_{k'=1}^{K} \frac{\eta_{lk'}}{M\gamma_{lk'l}} \mathbb{E}\left[\pmb{\hat g}_{lk'l}^T \ \pmb{\widetilde{g}}^*_{lkl} \ \pmb{\widetilde{g}}_{lkl}^T \ \pmb{\hat g}_{lk'l}^{*} \right] \nonumber \\
        &=& \rho_b(\beta_{lkl}-\gamma_{lkl})\left(\sum\limits_{k'=1}^{K}\eta_{lk'}\right)
\end{eqnarray}
The variance of coherent interference term in  in (\ref{eqn:completeot}) is given by
\begin{eqnarray}\label{eq:noiseF3}
	V_4^{BT} &\triangleq& \rho_b \sum_{\substack{j=1 \\ j \neq l}}^{L}\mathbb{E}\left[\Bigm| \pmb{g}_{lkj}^T \sqrt{\frac{\eta_{jk}}{M\gamma_{jkj}}} \ \pmb{\hat g}^*_{jkj} \ s_{jk} (t)\Bigm|^2\right] 
\end{eqnarray}
Substituting (\ref{eq:estimatehpll}) and (\ref{eqn:ylkp}) in (\ref{eq:noiseF3}), $V_4^{BT}$ can be written as
\begin{eqnarray}
V_4^{BT} &=& \rho_b\sum_{\substack{j=1 \\ j \neq l}}^{L}\frac{\eta_{jk}}{M\gamma_{jkj}}\alpha_{jk}^2\mathbb{E}\left[\left\vert \pmb{g}_{lkj}^T \ \left(\sqrt{\rho_p\tau_p}\pmb{g}^*_{lkj}
                          			 +\sum_{\substack{m=1 \\ m \neq l}}^{L} \sqrt{\rho_p\tau_p}\pmb{g}^*_{mkj}
             			   			 +\pmb{n}^*_{jk}\right) 
             			   	\ s_{jk} (t)\right\vert^2\right] \nonumber \\
         &=& \rho_b\sqrt{\rho_p\tau_p}\sum_{\substack{j=1 \\ j \neq l}}^{L}\frac{\eta_{jk}}{M\gamma_{jkj}}\alpha_{jk}^2\left(\mathbb{E}\left[\Bigm|\Bigm|\pmb{g}_{lkj}\Bigm|\Bigm|^4\right] + \sum\limits_{\substack{m=1 \\ m\neq l}}^{L}\mathbb{E}\Bigm[\pmb{g}^T_{mkj}\pmb{g}^*_{lkj}\pmb{g}^T_{lkj}\pmb{g}^*_{mkj}\Bigm]+\mathbb{E}\left[\Bigm|\Bigm|\pmb{g}_{lkj}\Bigm|\Bigm|^2\right]\right) \nonumber \\
    &=& M\rho_b\sum\limits_{\substack{j=1 \\ j \neq l}}^{L}\beta_{lkj}\eta_{jk}
\end{eqnarray}
The variance of the term transmitted for OT in (\ref{eqn:completeot}) is given by
\begin{eqnarray}
	V_5^{BT} &\triangleq& \rho_o\mathbb{E}\left[\Bigm|\sum_{j=1}^{L}\pmb{g}_{lkj}^T\Pi^{\perp}_{{\widehat{\pmb{G}}_j}}\pmb{U}_j \pmb{q}_{j}\Bigm|^2\right] 
	        = \rho_o\sum\limits_{j=1}^{L}\sum\limits_{i=1}^{L}\mathbb{E}\left[\pmb{q}_j^H\pmb{U}_j^H \Pi^{\perp}_{\pmb{\widehat{G}}_j} \pmb{g}_{lkj}^* \pmb{g}_{lki}^{T}\Pi^{\perp}_{\pmb{\widehat{G}}_i} \pmb{U}_i\pmb{q}_i\right] \nonumber \\
            &=&\rho_o\sum_{j=1}^{L}\sum_{i=1}^{L} \mathbb{E}\left[\pmb{g}_{lki}^{T}\Pi^{\perp}_{\pmb{\widehat{G}}_i}\pmb{U}_i\pmb{q}_i\pmb{q}^H_j\pmb{U}_j^H\Pi^{\perp}_{\pmb{\widehat{G}}_j}\pmb{g}_{lkj}^* \right] \nonumber
\end{eqnarray}
Since $\pmb{q}_j=\pmb{q}_i$, $\mathbb{E}[\pmb{q}_j^H\pmb{q}_i]=\frac{1}{M'}$. Thus
\begin{eqnarray}\label{eqn:noiseF4_1}	    
           V_5^{BT} &=&       \frac{\rho_o}{M'}\sum_{j=1}^{L}\sum_{i=1}^{L} \Tr\left(\mathbb{E}\left[\pmb{U}_i^H \Pi^{\perp}_{\pmb{\widehat{G}}_i} \pmb{g}_{lki}^* \pmb{g}_{lkj}^{T}\Pi^{\perp}_{\pmb{\widehat{G}}_j} \pmb{U}_j\right]\right)
\end{eqnarray}
When $i=j\neq l$, conditioned on $\pmb{\widehat{G}}_j$, $\pmb{U}_j$ is fixed so that
(\ref{eqn:noiseF4_1}) can be written as 
\begin{eqnarray}\label{eqn:v6m}
V_{5j}^{BT}&\triangleq& \frac{\rho_o}{M'}\sum\limits_{\substack{j=1 \\ j \neq l}}^{L} \Tr\left(\mathbb{E}\left[\pmb{U}_j^H \Pi^{\perp}_{\pmb{\widehat{G}}_j} \pmb{g}_{lkj}^* \pmb{g}_{lkj}^{T}\Pi^{\perp}_{\pmb{\widehat{G}}_j} \pmb{U}_j \right]\right) \nonumber \\
           &=& \frac{\rho_o}{M'}\sum_{\substack{j=1 \\ j \neq l}}^{L}\Tr\left(\mathbb{E}_{\pmb{\widehat{G}}_j}\left[\pmb{U}_j^H \Pi^{\perp}_{\pmb{\widehat{G}}_j} \mathbb{E}[\pmb{g}_{lkj}^* \pmb{g}_{lkj}^{T} | \pmb{\widehat{G}}_j]\Pi^{\perp}_{\pmb{\widehat{G}}_j} \pmb{U}_j \right]\right)
\end{eqnarray}
Since $\pmb{g}_{lkj}$ and $\pmb{\widehat{G}}_j$ are independent
\begin{eqnarray}\label{eqn:v6neql}
V_{5j}^{BT} = \rho_o\sum\limits_{\substack{j=1 \\ j \neq l}}^{L}\beta_{lkj}
\end{eqnarray}
Similarly when $i=j=l$, conditioned on $\pmb{\widehat{G}}_l$, $\pmb{U}_l$ is fixed but
\begin{eqnarray}\label{eqn:conditionghat}
E[\pmb{g}_{lkl}^*\pmb{g}_{lkl}^T|\pmb{\widehat{G}}_l] = \pmb{\hat g}_{lkl}^*\pmb{\hat g}_{lkl}^T + (\beta_{lkl}-\gamma_{lkl})\pmb{I}
\end{eqnarray}
Thus substituting (\ref{eqn:conditionghat}) and using the facts that $\Pi^{\perp}_{\pmb{\widehat{G}}_l}\pmb{\hat g}_{lkl} = 0$ and $\pmb{U}_l^H\pmb{U}_l=\pmb{I}$, (\ref{eqn:v6m}) can be evaluated to be
\begin{eqnarray}\label{eqn:v6l}
V_{5l}^{BT}= \rho_o(\beta_{lkl} -\gamma_{lkl})
\end{eqnarray}
For the scenario where $i\neq j$ the channel estimates of different cells are uncorrelated and $\mathbb{E}\left[\pmb{U}_j^H\Pi^{\perp}_{\pmb{\widehat{G}}_j}\Pi^{\perp}_{\pmb{\widehat{G}}_i} \pmb{U}_i\right]$ = $\mathbb{E}[\pmb{U}_j^H\pmb{U}_i] =0$. Thus
\begin{eqnarray}\label{eqn:v6}
   V_5^{BT} &=& V_{5l}^{BT} + V_{5j}^{BT} \nonumber \\ &=&\rho_o\left(\sum\limits_{j=1}^{L}\beta_{lkj}-\gamma_{lkl}\right)
\end{eqnarray}
The variance of the AWGN term is calculated as $1$.\\
Thus the information rate for the $k$-th BT of the $l$-th cell is given by
\begin{eqnarray}\label{eqn:rateotwitheta}
R_{lk}^{MR} &=& log_2\left[1+\dfrac{M\rho_b\eta_{lk}\gamma_{lkl}}{ \rho_b\sum\limits_{j=1}^{L}\beta_{lkj}\left(\sum\limits_{k'=1}^K\eta_{jk'}\right) +M\rho_b\sum\limits_{\substack{j=1 \\ j \neq l}}^{L}\beta_{lkj}\eta_{jk} +\rho_o\left(\sum\limits_{j=1}^{L}\beta_{lkj} -\gamma_{lkl}\right)+1}\right]
\end{eqnarray}
Applying the constraint (\ref{eqn:etaconstraint}), the value of rate is given by
\begin{eqnarray}\label{eqn:rateot}
R_{lk}^{MR} &=& log_2\left[1+\dfrac{M\rho_b\eta_{lk}\gamma_{lkl}}{ \rho_b\sum\limits_{j=1}^{L}\beta_{lkj}+M\rho_b\sum\limits_{\substack{j=1 \\ j \neq l}}^{L}\beta_{lkj}\eta_{jk} +\rho_o\left(\sum\limits_{j=1}^{L}\beta_{lkj} -\gamma_{lkl}\right)+1}\right]
\end{eqnarray}
%\begin{eqnarray}\label{eqn:finalratebt}
%R_{lk}^{MR} &=& log_2\left[1+\dfrac{M\rho_b\eta_{lk}\gamma_{lkl}}{\rho_b\sum\limits_{j=1}^{L}\beta_{lkj}+\rho_o\left(\sum\limits_{j=1}^{L}\beta_{lkj} -\gamma_{lkl}\right)+1}\right]
%\end{eqnarray}


\section{Power Allocation Stratergy for BT}
The max-min fairness power allocation is adopted as described in Chapter 5 and 6.2.5 in \cite{bib:MassiveMimoBook}. As per this policy the value of $\eta_{lk}$ shall be chosen so that the rates of all the BTs in a cell are equal. A heuristic algorithm, that corrects the power control coefficients to the first order of approximation for the effects of coherent interference is used.

From (\ref{eqn:rateot}), the value of $\pmb{SINR}_{lk}$ is
\begin{eqnarray}\label{eqn:sinrlkbt}
\pmb{SINR}_{lk} &=& \dfrac{M\rho_b\eta_{lk}\gamma_{lkl}}{ \rho_b\sum\limits_{j=1}^{L}\beta_{lkj}+M\rho_b\sum\limits_{\substack{j=1 \\ j \neq l}}^{L}\beta_{lkj}\eta_{jk} +\rho_o\left(\sum\limits_{j=1}^{L}\beta_{lkj} -\gamma_{lkl}\right)+1}
\end{eqnarray}
 The initial power control coefficients, $\hat \eta_{lk}$, is obtained under the assumption that the coherent interference term is negligible in each cell. From (\ref{eqn:sinrlkbt}) we obtain
 \begin{eqnarray}\label{eqn:etasinr}
 \hat \eta_{lk} &=& \frac{\pmb{SINR}_{lk}\left(\rho_b\sum\limits_{j=1}^{L}\beta_{lkj}+\rho_o\left(\sum\limits_{j=1}^{L}\beta_{lkj} -\gamma_{lkl}\right)+1\right)}{M\rho_b\gamma_{lkl}}
 \end{eqnarray}
 Using constraint (\ref{eqn:etaconstraint}) and substituting (\ref{eqn:sinrlkbt}) in (\ref{eqn:etasinr}), the value of  $\hat \eta_{lk}$ is given by
\begin{eqnarray}\label{eqn:etahatbt}
    \hat \eta_{lk} = \dfrac{\rho_b\sum\limits_{j=1}^{L}\beta_{lkj}+\rho_o\left(\sum\limits_{j=1}^{L}\beta_{lkj} -\gamma_{lkl}\right)+1}
    {\gamma_{lkl}\mathlarger{\mathlarger{\sum}}\limits_{k'=1}^{K}\cfrac{\rho_b\sum\limits_{j=1}^{L}\beta_{lk'j}+\rho_o\left(\sum\limits_{j=1}^{L}\beta_{lk'j} -\gamma_{lk'l}\right)+1}{\gamma_{lk'l}}}
\end{eqnarray}
Using $\hat \eta_{lk}$, the exact value of the signal to interference noise ratio, $\pmb{\widehat{SINR}}_{lk}^{MR}$, is obtained as
\begin{eqnarray}
\pmb{\widehat{SINR}}_{lk}^{MR} &=& \dfrac{M\rho_b\hat\eta_{lk}\gamma_{lkl}}{ \rho_b\sum\limits_{j=1}^{L}\beta_{lkj}+M\rho_b\sum\limits_{\substack{j=1 \\ j \neq l}}^{L}\beta_{lkj}\hat\eta_{jk} +\rho_o\left(\sum\limits_{j=1}^{L}\beta_{lkj} -\gamma_{lkl}\right)+1}
\end{eqnarray}
Then, within the limits of approximation, the quantity
\begin{eqnarray}\label{eqn:flk}
\hat f_{lk} = \frac{\pmb{\widehat{SINR}}_{lk}^{MR}}{\hat \eta_{lk}} 
\end{eqnarray}
may be interpreted as the rate that the $k$-th terminal in the $l$-th cell would obtain for $\eta_{lk}=1$. 
The value of $\hat f_{lk}$ is given by
\begin{eqnarray}
\hat f_{lk} &=& \dfrac{M\rho_b\gamma_{lkl}}{ \rho_b\sum\limits_{j=1}^{L}\beta_{lkj}+M\rho_b\sum\limits_{\substack{j=1 \\ j \neq l}}^{L}\beta_{lkj}\hat\eta_{jk} +\rho_o\left(\sum\limits_{j=1}^{L}\beta_{lkj} -\gamma_{lkl}\right)+1}
\end{eqnarray}
Thus for any $\eta_{lk}$ the resulting SINR for $k$-th BT in $l$-th cell is
\begin{eqnarray}\label{eqn:sinrlk}
\pmb{SINR}_{lk} &\approx & \left(\frac{\pmb{\widehat{SINR}}_{lk}^{MR}}{\hat \eta_{lk}}\right)\eta_{lk} \nonumber \\
                  &=& \hat f_{lk}\eta_{lk}   
\end{eqnarray}
By max-min fairness policy $\pmb{SINR}_{lk} = \overline{\pmb{SINR}}_{l} \ \forall \ k=1,\dotsc, K$. Using the constraint (\ref{eqn:etaconstraint}) in (\ref{eqn:sinrlk}), the value of $\overline{\pmb{SINR}}_{l}$ is given by
\begin{eqnarray}\label{eqn:sinrbar}
\overline{\pmb{SINR}}_{l} &=& \cfrac{1}{\sum\limits_{k=1}^K\cfrac{1}{\hat f_{lk}}}          =\frac{M\rho_b}{\mathlarger{\mathlarger{\sum}}\limits_{k'=1}^{K}\cfrac{\rho_b\sum\limits_{j=1}^{L}\beta_{lk'j}+\rho_o\left(\sum\limits_{j=1}^{L}\beta_{lk'j} -\gamma_{lk'l}\right)+M\rho_b\sum\limits_{\substack{j=1 \\ j \neq l}}^{L}\beta_{lk'j}\hat\eta_{jk'}+1}{\gamma_{lk'l}}}
\end{eqnarray}
 and the value of $\eta_{lk}$ is given by
\begin{eqnarray}\label{eqn:etalkgen}
\eta_{lk} &=& \frac{\overline{\pmb{SINR}}_{l}}{\hat f_{lk}}  = \cfrac{\cfrac{1}{\hat f_{lk}}}{\sum\limits_{k=1}^K\cfrac{1}{\hat f_{lk}}} =\frac{\rho_b\sum\limits_{j=1}^{L}\beta_{lkj}+\rho_o\left(\sum\limits_{j=1}^{L}\beta_{lkj} -\gamma_{lkl}\right)+M\rho_b\sum\limits_{\substack{j=1 \\ j \neq l}}^{L}\beta_{lkj}\hat\eta_{jk}+1}{\gamma_{lkl}\mathlarger{\mathlarger{\sum}}\limits_{k'=1}^{K}\cfrac{\rho_b\sum\limits_{j=1}^{L}\beta_{lk'j}+\rho_o\left(\sum\limits_{j=1}^{L}\beta_{lk'j} -\gamma_{lk'l}\right)+M\rho_b\sum\limits_{\substack{j=1 \\ j \neq l}}^{L}\beta_{lk'j}\hat\eta_{jk'}+1}{\gamma_{lk'l}}}
\end{eqnarray}

\section{Performance for the O-Terminals}
	The signal received by OT is a combination of data to the BT, data for the OT and the noise. We assume that the OT experiences independent Rayleigh fading such that the channel response from the BS of $j$-th cell to OT of $l$-th cell is $\pmb{h}_{jl}$ $\sim CN(0,\pmb{C}_{\pmb{h}_{jl}}$) where $\pmb{C}_{\pmb{h}_{jl}} = \beta_{ojl}I$. To find the achievable rates of transmission, the channel from the BS to the OT shall be estimated and using this estimated value the received signal at the OT shall be evaluated to find the signal power, noise/interference variance.

\subsection{Pilot Phase}
    We consider a modified version of JBB, named as JBB', in which we assume there is no transmission to BT during the pilot phase of OT. 
    
    The signal received by OT in the $l$-th cell, in the pilot phase, is given by
\begin{eqnarray}
	y_{ol}(t)=\sqrt{\rho_o} \sum_{j=1}^{L} \pmb{h}^T_{ejl} \ \pmb{q}^p(t) + w_{ol}(t) 
\end{eqnarray}
where
\begin{eqnarray}
	\pmb{h}_{ejl} &\triangleq& \pmb{U}_j^T\Pi^{\perp}_{{\widehat{\pmb{G}}^T_j}}\pmb{h}_{jl} = \pmb{U}_j^T\pmb{h}_{jl} \\
    \mathbb{E}[\pmb{h}_{ejl}\pmb{h}^H_{ejl}|\pmb{\widehat{G}}_j] &=& \beta_{ojl}\pmb{I} \nonumber \\
                                           &\triangleq& \pmb{C}_{\pmb{h}_{ejl}} \nonumber
\end{eqnarray}
and $\pmb{q}^p(t)$ are the known pilot signals sent at time $t$.
The OT correlates $y_{ol}(t)$ with the pilot sequence to obtain
\begin{eqnarray}
	\pmb{y}_{ol}^{p} &\triangleq& \sum_{t=1}^{\tau_p^o} \Big(\pmb{q}^{p}(t)\Big)^* y^T_{ol}(t)\nonumber
\end{eqnarray}
where $\tau_p^o$ is the total number of symbols send in the pilot phase.

The total power transmitted for the broadcast of SI is the same during pilot and payload phase. Therefore
\begin{eqnarray}
    \sum\limits_{t=1}^{\tau_p^o}\Big(\pmb{q}^p_{l}(t)\Big)^*\Big(\pmb{q}^p_{l}(t)\Big)^T&=&\frac{\tau_p^o}{M'}\pmb{I} \\
\implies \pmb{y}_{ol} &=& \sqrt{\rho_o}\frac{\tau_p^o}{M'}\sum_{j=1}^{L}\pmb{h}_{ejl}+\pmb{n}^p_{ol}
\end{eqnarray}
The variance of AWGN, $\pmb{n}_{ol}^p$, is given by
\begin{eqnarray}
	\mathbb{E}[\pmb{n}^p_{ol}(\pmb{n}^p_{ol})^H] &=& \frac{\tau_p^o}{M'}I \nonumber
\end{eqnarray}
Following the same procedure as in section (\ref{sec:btchesti}), the channel estimate from the $l$-th cell BS to the OT can be derived as 
\begin{eqnarray}\label{eqn:otchesti}
	\pmb{\hat h}_{ell} &=& \frac{M'\sqrt{\rho_o}\beta_{oll}}{M'+\rho_o\tau_p^o\sum\limits_{j=1}^{L}\beta_{ojl}}\pmb{y}^p_{ol}
\end{eqnarray}
Similarly the co-variance matrix of the estimate is
\begin{eqnarray}
	\pmb{C}_{\pmb{\hat h}_{ell}} &\triangleq& \mathbb{E}[\pmb{\hat h}_{ell}\pmb{\hat h}^H_{ell}] = \frac{\rho_o\tau_p^o\beta^2_{oll}}{\rho_o\tau_p^o\sum\limits_{j=1}^{L}\beta_{ojl}+M'}\pmb{I}
\end{eqnarray}
and the co-variance matrix of error in estimate is
\begin{eqnarray}
	\pmb{C}_{\pmb{\widetilde{h}}_{ell}} &=& \beta_{oll}\pmb{I} - 	\pmb{C}_{\pmb{\hat h}_{ell}}
\end{eqnarray}
\subsection{Payload Phase}

The received signal at OT in the $l$-th cell is given by
\begin{eqnarray}\label{eqn:otytorg}
y_{ol}(t) = \sqrt{\rho_o} \sum_{j=1}^{L} \pmb{h}^T_{ejl} \pmb{q}_j(t)
+ \sqrt{\rho_b} \sum_{j=1}^{L} \pmb{h}^T_{jl} \left(\sum_{k'=1}^{K} \pmb{v}_{jk'} s_{jk'}(t)\right) + w_{ol}(t)  
\end{eqnarray} 
Since there is no transmission to BTs during the pilot phase of OT, in order to spend the same amount of energy per coherence interval, $\rho_b$ in $(\ref{eqn:otytorg})$ must be replaced with
\begin{eqnarray}
\rho_b' \triangleq \frac{\tau_d^d}{\tau_d^d-\tau_p^o}\rho_b
\end{eqnarray}
where $\tau_d^d$ is the total number of downlink payload symbols.
Thus the received signal is
\begin{eqnarray}\label{eqn:otyt}
 y_{ol}(t) = \sqrt{\rho_o} \sum_{j=1}^{L} \pmb{h}^T_{ejl} \pmb{q}_j(t)
            + \sqrt{\rho_b'} \sum_{j=1}^{L} \pmb{h}^T_{jl} \left(\sum_{k'=1}^{K} \pmb{v}_{jk'} s_{jk'}(t)\right) + w_{ol}(t)  
\end{eqnarray} 
Substituting $\pmb{h}_{ell} \triangleq \pmb{\hat h}_{ell}-\pmb{\widetilde{h}}_{ell}$, equation (\ref{eqn:otyt}) can be written as
\begin{eqnarray}\label{eqn:otcomplete}
  y_{ol}(t)= \sqrt{\rho_o} \pmb{\hat h}^T_{ell} \ \pmb{q}_l(t) 
	    - \sqrt{\rho_o}\pmb{\widetilde{h}}^T_{ell}\pmb{q}_l(t)
	    +\sqrt{\rho_o} \sum_{\substack{j=1 \\ j \neq l}}^{L} \pmb{h}^T_{ejl}\pmb{q}_{j}(t) 
            + \sqrt{\rho_b'} \sum_{j=1}^{L} \pmb{h}^T_{jl} \left(\sum_{k'=1}^{K} \pmb{v}_{jk'} s_{jk'}(t) \right)             
	    + w_{ol}(t)  
\end{eqnarray}
The first term in (\ref{eqn:otcomplete}) is the required signal and other terms constitute to interference and noise. The signal, intereference and the noise terms are mutually uncorrelated. 

The signal power is given by
\begin{eqnarray}
SP^{OT} &\triangleq& \rho_o\mathbb{E}\left[\Bigm|\pmb{\hat h}_{ell}^T \pmb{q}_l(t)\Bigm|^2 | \pmb{\hat h}_{ell}\right] = \frac{\rho_o}{M'}\mathbb{E}\left[\Bigm|\Bigm|\pmb{\hat h}^T_{ell}\Bigm|\Bigm|^2 | \pmb{\hat h}_{ell}\right] \nonumber \\    
         &=& \frac{\rho_o}{M'} \Bigm|\Bigm|\pmb{\hat h}_{ell}\Bigm|\Bigm|^2
\end{eqnarray}
The noise variances for the other terms in (\ref{eqn:otcomplete}) are as follows
\begin{eqnarray}\label{eqn:otv1}
	V_1^{OT} &\triangleq=& \rho_o\mathbb{E}\left[\Bigm|\pmb{\widetilde{h}}_{ell}^T \pmb{q}_l(t)\Bigm|^2|\pmb{\hat h}_{ell}\right] \nonumber \\
            &=& \frac{\rho_o}{M'} \mathbb{E}\left[\Bigm|\Bigm|\pmb{\widetilde{h}}^T_{ell}\Bigm|\Bigm|^2 |\pmb{\hat h}_{ell}\right] \nonumber \\
            &=& \rho_o\left(\beta_{oll}-\frac{\rho_o\tau_p^o\beta^2_{oll}}{\rho_o\tau_p^o\sum\limits_{j=1}^{L}\beta_{ojl}+M'}\right)          
\end{eqnarray}
\begin{eqnarray}\label{eqn:otv2}
	V_2^{OT} &\triangleq& \rho_o \mathbb{E}\left[\left\vert\sum_{\substack{j=1 \\ j \neq l}}^{L} \pmb{h}^T_{ejl} \pmb{q}_j(t)\right\vert^2|\pmb{\hat h}_{ell}\right] \nonumber \\
            &=& \rho_o\sum_{\substack{j=1 \\ j\neq l}}^{L} \beta_{ojl}
\end{eqnarray}
\begin{eqnarray}\label{eqn:v3ot}
	V_3^{OT} &\triangleq& \rho_b' \mathbb{E}\left[\left| \sum_{j=1}^{L}\sum_{k'=1}^{K} \pmb{h}^T_{jl} \pmb{v}_{jk'}s_{jk'}(t) \right|^2 | \pmb{\hat h}_{ell} \right] 
\end{eqnarray}
When $i=j=l$,
\begin{eqnarray}
     V_{3l}^{OT} &\triangleq& \rho_b' \mathbb{E}\left[\sum_{k'=1}^{K} \pmb{v}_{lk'}^H \pmb{h}_{ll}^*  \pmb{h}_{ll}^T\pmb{v}_{lk'}| \pmb{\hat h}_{ell} \right] \nonumber \\
            &=& \rho_b' \mathbb{E}_{\pmb{\widehat{G}}_l}\left[\mathbb{E}\left[\sum_{k'=1}^{K} \pmb{v}^H_{lk'} \pmb{h}_{ll}^* \pmb{h}_{ll}^T \pmb{v}_{lk'}| \pmb{\hat h}_{ell},  \pmb{\widehat{G}}_{l} \right] | \pmb{\hat h}_{ell}\right] \nonumber \\
            &=& \rho_b' \mathbb{E}_{\pmb{\widehat{G}}_l}\left[\sum_{k'=1}^{K}\pmb{v}^H_{lk'} \mathbb{E}\left[ \pmb{h}_{ll}^*  \pmb{h}^T_{ll} | \pmb{\hat h}_{ell},  \pmb{\widehat{G}}_{l} \right] \pmb{v}_{lk'}| \pmb{\hat h}_{ell}\right]
\end{eqnarray}
The inner expectation can simplified using the equality from \cite{bib:rmtBook}.
\begin{eqnarray}
	\mathbb{E}[\pmb{h}_{ll}^*\pmb{h}^T_{ll}|\pmb{\hat h}_{ell}, \pmb{\widehat{G}}_{l}] = \pmb{C}_{\pmb{h}_{ll}}-\mathbb{E}[\pmb{h}_{ll}^* \pmb{\hat h}^T_{ell}| \pmb{\widehat{G}}_{l}] \ . \ \pmb{C}_{\pmb{\hat h}_{ell}}^{-1} .\ \mathbb{E}[\pmb{\hat h}_{ell}^* \pmb{h}_{ll}^T| \pmb{\widehat{G}}_{l}]
\end{eqnarray}
where
\begin{eqnarray}
	\mathbb{E}[\pmb{h}_{ll}^* \pmb{\hat h}_{ell}^T | \pmb{\widehat{G}}_{l}] = \frac{\beta_{oll}^{2}\rho_o\tau_p^o}{M'+\rho_o\tau_p^o\sum\limits_{j=1}^{L}\beta_{ojl}}\pmb{U}_l
\end{eqnarray}
Also note that $\pmb{U_l}^H\pmb{v}_{lk}=0$ for all $k$ since $\pmb{U}_l^H\pmb{\widehat{G}}_l=0$. Thus
\begin{eqnarray}\label{eqn:otv3}
	V_{3l}^{OT} &=& \rho_b'\left( \beta_{oll} \mathbb{E}\left[\sum_{k'=1}^{K}||\pmb{v}_{lk'}||^2\right]
					- \frac{\beta_{oll}^{2}\rho_o\tau_p^o}{M'+\rho_o\tau_p^o\sum\limits_{j=1}^{L}\beta_{ojl}} \mathbb{E}\left[\sum_{k'=1}^{K} \pmb{v}_{lk'}^H \pmb{U}_l \pmb{U}_l^H \pmb{v}_{lk'}\right]\right) \nonumber \\
%	    &=& \rho_b'\sum_{j=1}^{L}\beta_{oj}
        &=& \rho_b'\beta_{oll}
\end{eqnarray}
When $i=j\neq l$
\begin{eqnarray}\label{eqn:v3jot}
V_{3j}^{OT} &\triangleq& \rho_b' \sum_{\substack{j=1 \\ j\neq l}}^{L} \mathbb{E}\left[\sum_{k'=1}^{K} \pmb{v}_{jk'}^H \pmb{h}_{jl}^*  \pmb{h}_{jl}^T \pmb{v}_{jk'}| \pmb{\hat h}_{ell} \right] \nonumber \\
           &=&  \rho_b' \sum_{\substack{j=1 \\ j\neq l}}^{L}\left(\sum_{k'=1}^{K} \mathbb{E}\left[||\pmb{v}_{jk'}||^2\right]\right)\mathbb{E}\left[ \Bigm|\Bigm|\pmb{h}_{jl}\Bigm|\Bigm|^2\right] \nonumber \\
           &=&\rho_b'M' \sum_{\substack{j=1 \\ j\neq l}}^{L}\beta_{ojl}
\end{eqnarray}
When $i\neq j\neq l$ the value is zero. Thus
\begin{eqnarray}
V_3^{OT} &=&\rho_b' \left(\beta_{oll}+ M' \sum_{\substack{j=1 \\ j\neq l}}^{L}\beta_{ojl}\right)
\end{eqnarray}

For the AWGN term, the noise variance is, $V_4^{OT} = 1$

The rate for the OT is
\begin{eqnarray}\label{eqn:rateotfull}
	R_{ol} &\triangleq& \mathbb{E}_{\pmb{\hat h}_{ell}}\left[log_2\left(1+\frac{\frac{\rho_o||\pmb{\hat h}_{ell}||^2}{M'}}{V_1^{OT}+V_2^{OT}+V_3^{OT}+V_4^{OT}}\right) \right] \nonumber \\
            &=&  \mathbb{E}_{\pmb{\hat h}_{ell}}\left[log_2\left(1+\frac{\frac{\rho_o||\pmb{\hat h}_{ell}||^2}{M'}}
								 {\rho_o\left(\beta_{oll}-\frac{\rho_o\tau_p^o\beta^2_{oll}}{\rho_o\tau_p^o\sum\limits_{j=1}^{L}\beta_{ojl}+M'}\right)        
								  + \rho_o \sum\limits_{\substack{j=1 \\ j\neq l}}^{L} \beta_{ojl} 
								  +\rho_b' \left(\beta_{oll}+ M' \sum\limits_{\substack{j=1 \\ j\neq l}}^{L}\beta_{ojl}\right)
								  +1}\right) \right]	
\end{eqnarray}
To obtain a simple closed-form bound of the expectation in (\ref{eqn:rateotfull}), from \cite{bib:dtsysBook}, if $\pmb{\Psi}$ is an $M'$-vector with independent ${\mathbb C}{\mathbb N}(0,1)$ elements, then for any $\alpha>0$
\begin{eqnarray}\label{eqn:bound}
	\mathbb{E}[log_2(1+\alpha||\pmb{\Psi}||^2)] &\geq& log_2\left(1+\frac{\alpha}{E[\frac{1}{||\pmb{\Psi}||^2}]}\right) \nonumber \\
                                              &=&  log_2(1+(M'-1)\alpha)
\end{eqnarray}
Using the value of the variance of the estimate, bound for the rate can be formulated as
\begin{eqnarray}
	R_o &\geq& \left[log_2\left(1+\frac{\frac{\rho_o(M'-1)}{M'}\frac{\rho_o\tau_p^o\beta^2_{oll}}{\rho_o\tau_p^o\sum\limits_{j=1}^{L}\beta_{ojl}+M'}}
								 {\rho_o\left(\beta_{oll}-\frac{\rho_o\tau_p^o\beta^2_{oll}}{\rho_o\tau_p^o\sum\limits_{j=1}^{L}\beta_{ojl}+M'}\right)        
                                     + \rho_o \sum\limits_{\substack{j=1 \\ j\neq l}}^{L} \beta_{ojl} 
                                     +\rho_b' \left(\beta_{oll}+ M' \sum\limits_{\substack{j=1 \\ j\neq l}}^{L}\beta_{ojl}\right)
                                     +1}\right) \right]	
\end{eqnarray}
\bibliography{jbb}{}
\bibliographystyle{plain}
\enddocument
