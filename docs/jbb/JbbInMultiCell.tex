% !TeX spellcheck = en_US 
\documentclass[10pt, a4paper, twoside,fleqn]{article}
\usepackage[left=2cm, right=2cm, top=1.8cm]{geometry}
\usepackage[utf8]{inputenc}
\usepackage[english]{babel}
\usepackage{cite}
\usepackage[title]{appendix}
\usepackage{graphicx}
\usepackage{color}
\usepackage{amsmath}
\usepackage{amssymb}
\usepackage{relsize}
\usepackage{graphicx}  
\usepackage{subcaption}
\usepackage{url}
\usepackage{float}
\usepackage{mathtools}
\usepackage{bm}
% When building with pdftex this will hyperref with dvipdfm option will not work. 
% Uncomment the statement if building with latex->dvi2pdf
%\usepackage[dvipdfm]{hyperref}

\graphicspath{{./fig/}}

\title{Joint Beamforming and Broadcasting in Massive MIMO for Multicell Systems}
\author{}
\date{April 2018}
 
\begin{document}
\maketitle
%\begin{titlepage}
%   \vspace*{\stretch{1.0}}
%   \begin{center}
%      \Large\textbf{Joint Beamforming and Broadcasting in Massive MIMO for Multicell systems}\\
%      \large\textit{A. Thor}
%   \end{center}
%   \vspace*{\stretch{2.0}}
%\end{titlepage}
 
\section{Introduction}
	Joint Beamforming and Broadcasting(JBB) is a novel idea proposed in \cite{bib:jbb} in which data and the system information is sent simultaneously to different terminals in a cell. The cell to which data is beamformed is called as B-terminals and others are called as O-terminals. The transmission works on the principle that if the data to the O-terminals is sent in the nullspace of channel estimate of the B-terminals then the interference to B-terminals is negligible. The paper \cite{bib:jbb} deals only with a single cell case. In this document we are trying to extend it to multicell case.
\footnote{The document is mainly for review purpose. The person reading this document is assumed to have a good knowledge on Joint Beamforming and Broadcasting or have read the paper \cite{bib:jbb} along with understanding on massive MIMO multi cell systems.}
	
	The aim of the document is to find the achievable rates of transmission for B-terminals along with its power allocation scheme. For the O-terminals we try to bound the rate expression. In the document the following shall be evaluated. For the B-terminals, under the JBB in multi cell scenario, the channel from the base station to the terminals will be estimated. This estimation will be subjected to pilot contamination. Once the channel is estimated the performance of the B-terminals, under JBB. The signal power and the noise powers are evaluated to come up with the achievable rate. For O-terminals, the channel estimation is done during the pilot phase. In the payload phase the signal received at an O-terminal shall be evaluated to get the signal power and noise variances.

	The document tries to follow the notation available in \cite{bib:MassiveMimoBook}. The explanation of notations of all the symbols used are provided in Appendix \ref{app:notation}. Some of the results which are derived or directly stated are already available in \cite{bib:MassiveMimoBook}.

\section{Approach and Assumptions}	
	In any particular cell the data is sent to B-terminals and O-terminals simultaneously. Also all the base stations transmit the data to their respective terminals at the same time. The base station knows the channel estimate of B-terminals but the channel estimate is not known for the O-terminals. So there is a pilot phase for the O-terminals where the base station sends the pilot signals for the O-terminals to estimate its channel. We assume that during this pilot phase there is no transmission happening for the B-terminals to estimate their own channel.

	A multicell system with $L$ cells each cell having $K$ B-terminals which uses the same frequency band are assumed. Each base station is assumed to have $M$ antennas. The transmission from the base station or from the terminals are synchronous both in pilot phase and payload phase. This is to take care off the worst case scenario. The $l^{th}$ base station transmits the $M$-vector
\begin{eqnarray}
	\pmb{x}(t) = \sqrt{\rho_b}\sum\limits_{k'=1}^{K}\boldsymbol{v}_{lk'}s_{lk'}(t)
    		      + \sqrt{\rho_o}\Pi^{\perp}_{{\widehat{G}_l}}
\end{eqnarray}
where $\pmb{v}_{lk'}$ and $s_{lk'}(t)$ are the beamforming vectors and information symbols for the $k'^{th}$ user terminal (UT) in the $l^{th}$ cell, $z_l(t)$ is the system information containing vector intended for the O-terminals and $\Pi^{\perp}_{{\widehat{G}_l}}$ denotes the projection onto to the space orthogonal to the column space of $\hat G_l$. The symbols, $s_{lk'}(t)$ are assumed to have zero mean and unit variances, and the beamforming vectors $\pmb{v}_{lk'}$ depends on the estimate of the channel fain from UTs in the $l^{th}$ cell to the BS of $l^{th}$ cell. 
The beamforming vectors also satisfy the criterion
\begin{eqnarray}
	\pmb{E}\left[||\sum\limits_{k'=1}^{L}\pmb{v}_{lk'}||^2\right]=1.
\end{eqnarray}
and $z_l(t)$ is normalized such that
\begin{eqnarray}\label{eqn:zlcondition}
	\pmb{E}\left[||\Pi^{\perp}_{{\widehat{G}_l}}z_l(t)||^2\right]=1.
\end{eqnarray}
The O-terminal sees $z_l(t)$ transmitted over the channel with effective channel response $\Pi_{\frac{1}{\pmb{\hat G}_l}}\pmb{h}_l$ which is unknown to the O-terminal. We assume that $z_l(t)$ is confined to the subspace of dimension $M'$, where $M'\leq M$ and $z(t) = \pmb{U}_l\pmb{q}_l(t)$. The vector $\pmb{q}_l(t)$ are the information symbols relating to the system information (SI) and are iid $CN(0,\zeta)$ distributed. From \cite{bib:jbb} it follows that $\zeta=\frac{1}{M'}$ contains the information for a O-terminals  and is assumed to have independent $CN(0, \zeta)=\frac{1}{M'}$. $\pmb{U}_l$ is a $M\times M'$ semi-unitary matrix which is unknown to the O-terminals. Under independent Rayleigh fading we assume $\pmb{U}_l^H\pmb{U}_l=I$ and $\Pi_{\frac{1}{\pmb{\hat G}_l}}\pmb{U}_l=\pmb{U}_l$ same as in \cite{bib:jbb}. In addition, without loss of generality, it is assumed that $\pmb{U}_j^H\pmb{U}_l=0$.

The additive white gaussian noise (AWGN) is assumed to be $CN(0,1)$ and is independent of the channel, its estimate and the transmitted symbols.

\section{Performance for the B-terminals}
	In this section performance of the JBB for the B-terminals in multicell scenario is evaluated. The section aims at finding the achievable rate for the B-terminals. Since channel estimate, its variance and the error variance of the estimate plays an important role in finding the rate we start with estimation of channel for a particular terminal in a specific cell. The variance in estimation and error are then calculated. From the expression of the received signal at any particular B-terminal in a specific cell, the desired signal power and the variance of noise and interference terms are calculated to get the expression for hte signal to interference and noise ratio (SINR) and the corresponding information rate.

\subsection{Estimation of channel for B-terminal} \label{sec:btchesti}
	The channel is estimated during the pilot phase in uplink. The B-terminals send the known pilots to the base station and base station estimates the channel. Here we are assuming that same set of orthogonal pilots are used in all the cells and all the $K$ terminals in each of the $L$ cells send their pilots simultaneously to their respective base stations which results in pilot contamination. The approach followed is similar to that of Chapter 4 in \cite{bib:MassiveMimoBook}

The signal received by the $l^{th}$ base station due to the simultaneous pilot transmission from  all the $K$ terminals of $L$ cells is given by
\begin{eqnarray}\label{eq:rcvy}
	\pmb{y}_{l} &=& \sum_{j=1}^{L} \sum_{k'=1}^{K} (\sqrt{\rho_p \tau_p \ \beta_{jk'l}} \ \pmb{g}_{jk'l} \ \pmb{\phi}^H_{jk'})
        		    +  \pmb{n}_{l} \nonumber \\                    
    			   &=& \sqrt{\rho_p \tau_p \beta_{lkl}} \ \pmb{g}_{lkl} \ \pmb{\phi}^H_{lk}
        			+  \sum_{\substack{k'=1 \\ k' \neq k}}^{L} \sqrt{\rho_p \tau_p \beta_{lk'}^{l}} \ \pmb{g}_{lk'l} \ \pmb{\phi}^H_{lk'}
        		    +  \sum_{\substack{j=1 \\ j \neq l}}^{L} \sum_{\substack{k'=1 \\ k' \neq k}}^{K} \sqrt{\rho_p \tau_p \beta_{jk'}^{l}} \
                       \pmb{g}_{jk'l} \ \pmb{\phi}^H_{lk'}
    			    +  \pmb{n}_{l}
\end{eqnarray}
where $g_{j'k'l} \in {\mathbb C}^{M \times 1}$  and $\beta_{j'k'l}$ are the fast fading components of the channel vector and the path loss respectively between the $k'^{th}$ UT in the $j'^{th}$ cell and the $l^{th}$ BS. The elements of $g_{j'k'l}$ has $CN(0,1)$ i.i.d distribution. The pilot sequence transmitted by the $k'^{th}$ UT in the $l'^{th}$ cell is denoted by $\phi_{lk} \in {\mathbb C}^{\tau_p \times 1}$ where $\tau_p$ denotes the length of the pilot sequence. The sequences are orthonormal, i.e., $\pmb{\phi}_{j'q}^H \pmb{\phi}_{jq} = 0$ for  $q \ne q'$ and is $1$ for $q=q'$.

For estimating the channel of the $k^{th}$ terminal of the $l^{th}$ cell, at the BS of the $l^{th}$ cell the received vector $y_{l}$ is multiplied by the known pilot $\pmb{\phi}_{lk}$. Since the pilots are orthogonal, only those terms remain which have the pilots sent by the $k^{th}$ terminal of each of cell. Thus (\ref{eq:rcvy}) can be written as
\begin{eqnarray}
	\pmb{y}_{l}^{p} \triangleq \pmb{y}_{l} \pmb{\phi}_{lkl}
                          &=&     \sqrt{\rho_p\tau_p\beta_{lkl}}\pmb{g}_{lkl}
                           +      \sum_{\substack{j'=1 \\ j' \neq l}}^{L} \sqrt{\rho_p\tau_p\beta_{jkl}}\pmb{g}_{jkl}
             			   +      \pmb{n}_{l}\pmb{\phi}_{lkl}
\end{eqnarray}

Let the MMSE estimate of the channel $\pmb{g}_{lkl}$ be $\pmb{\hat g}_{lkl}$. Thus $\pmb{\hat g}_{lkl} = \alpha_{lk}\pmb{y}_{l}^{p}$. The value of $\alpha_{lk}$ shall be chosen such that
        
\begin{eqnarray}
	\pmb{E}[(\pmb{g}_{lkl}-\alpha_{lk}\pmb{y}_{l}^{p})^H \pmb{y}_{l}^{p}] = 0 \\
	\alpha_{lk} = \frac{\pmb{E}[\pmb{g}^{H}_{lkl} \pmb{y}_{l}^{p}]}{\pmb{E}[|\pmb{y}_{l}^{p}|^2]}
\end{eqnarray}
Substituting the values for $\pmb{g}_{lpl}$ and $\pmb{y}_{l}^{p}$ in the above equation and taking the required expectations
\begin{eqnarray}\label{eq:alphapll}
	\alpha_{lk}=\frac{\sqrt{\rho_p \tau_p \beta_{lkl}}}{1+\rho_p\sum\limits_{j=1}^{L}\beta_{jkl}}
\end{eqnarray}
and the estimate of the channel from the $k^{th}$ terminal of the $l^{th}$ cell to the $l^{th}$ BS is given as
\begin{eqnarray}\label{eq:estimatehpll}
	\pmb{\hat g}_{lp}^{l} = \frac{\sqrt{\rho_p\tau_p\beta_{lkl}}}{1+\rho_p\sum\limits_{j=1}^{L}\beta_{jkl}} \pmb{y}_{l}^{p}
\end{eqnarray}
Let the error in the estimate be ${\pmb{\widetilde{g}}_{lp}^{l}}$ and the error is assumed to be uncorrelated with the $\pmb{\hat g}_{lp}^{l}$. Also $\pmb{\hat g}_{lp}^{l}=\pmb{g}_{lp}^{l}+\pmb{\widetilde{g}}_{lp}^{l}$.
The variance of the estimate and that of the error can be calculated as
\begin{eqnarray}
	\gamma_{lk} \triangleq \pmb{E}[\pmb{\hat g}_{lk}\pmb{\hat g}^H_{lk}]
                    =       \frac{\rho_p\tau_p\beta^2_{lkl}I}{1+\rho_p\sum\limits_{j=1}^{L}\beta_{jkl}} 
\end{eqnarray}
\begin{eqnarray}
    \pmb{E}[\pmb{\widetilde{g}}_{lk}^{l} \pmb{\widetilde{g}}^H_{lkl}] &=& \beta_{lkl}- \frac{\rho_p\tau_p\beta^2_{lkl}I}{1+\rho_p\sum\limits_{j=1}^{L}\beta_{jkl}} \nonumber\\
          									    &=& (\beta_{lkl} - \gamma_{lk})I
\end{eqnarray}

\subsection{Achievable Rate for B-terminals}

The data received at the $k^{th}$ terminal of the $l^{th}$ cell is given by
\begin{eqnarray}\label{eq:ot}
 	y_{l}(t) = \sqrt{\rho_b}\sum_{j=1}^{L} \sqrt{\beta_{lkj}} \pmb{g}_{lkj}^{T} \left(\sum_{k'=1}^{K}\pmb{v}_{jk'}s_{jk'}(t)\right)
 		      + \sqrt{\rho_o}\sum_{j=1}^{L}\sqrt{\beta_{lkj}}\pmb{g}^T_{lkj} \Pi^{\perp}_{{\widehat{G}_j}} \pmb{U}_j\pmb{q}_{j} \nonumber \\
 		      + \pmb{w}_{l}(t)
\end{eqnarray}  
where  $\pmb{g}_{lkj}$ is the channel response having $CN(0,\beta_{lkj})$ elements. The value of $\pmb{g}_{lkj}$ can be written as
\begin{eqnarray}\label{eqn:gvect}
 \pmb{g}_{lkj} =  \pmb{\hat g}_{lkj} - \pmb{\widetilde{g}}_{lkj}
\end{eqnarray}
where  $\pmb{\widetilde{g}}_{lkj}$ is the error in the estimate.

The values of beamforming vectors $\pmb{v}_{jk}$ depends on the channel estimate $\pmb{g}_{jkj}$. Considering Maximum Ratio Transmission, the value of the beamforming vectors the $k^{th}$ UT in the $j^{th}$ cell is given by
\begin{eqnarray}\label{eq:bfv}
	\pmb{v}_{jk} = \sqrt{\frac{\eta_{jk}}{M\gamma_{jkj}}}\pmb{\hat g}^*_{jkj}
\end{eqnarray}
where power control parameters, $\eta_{jk}$, satisfy 
\begin{eqnarray}
\sum\limits_{k=1}^{k=K}\eta_{jk}=1
\end{eqnarray}
The received signal at the $k^{th}$ B-terminal is given by
\begin{eqnarray}\label{eqn:rxbt}
    y_{lk}(t) = \sqrt{\rho_b}\sum\limits_{j=1}^{L} \ \pmb{g}_{lkl}^T \left( \sum\limits_{k'=1}^{k'=K}\pmb{v}_{jk'} \ s_{jk'}(t)\right) + \sqrt{\rho_o}\sum_{j=1}^{L}\pmb{g}_{lkj} \Pi_{\frac{1}{\hat G_j}} \pmb{U}_j \pmb{q}_{j}
    + w_{lk}(t) 
\end{eqnarray}
Substituting (\ref{eq:bfv}), (\ref{eqn:gvect}) in (\ref{eqn:rxbt}) and considering the term for the $k^{th}$ terminal in the $l^{th}$ cell, (\ref{eq:ot}) can be rewritten as

\begin{eqnarray} \label{eqn:completeot}
	y_{l}(t)  &=& \sqrt{\frac{\rho_b\eta_{lk}}{M\gamma_{lkl}}} \ \pmb{E}[|| \ \pmb{\hat g}_{lkl}||^2]s_{lk}(t)
	           +  \sqrt{\frac{\rho_b\eta_{lk}}{M\gamma_{lk}}}\left(||\pmb{\hat g}_{lkl}||^2 - \pmb{E}[||\pmb{\hat g}_{lkl}||^2]\right)s_{lk}(t) \nonumber \\    
               &+& \sqrt{\rho_b} \ \pmb{\hat g}_{lkl}^T \left(\sum_{\substack{k'=1 \\ k' \neq k}}^{K} \sqrt{\frac{\eta_{jk'}}{M\gamma_{lk'}}} \ \pmb{\hat g}^*_{lk'l} \ s_{lk'l} (t)\right) 
               + \sqrt{\rho_b} \sum_{\substack{j=1 \\ j \neq l}}^{L} \pmb{\hat g}_{lkl}^T \left(\sum_{k'=1}^{K} \sqrt{\frac{\eta_{jk'}}{M\gamma_{jk'}}} \ \pmb{\hat g}^*_{jk'j} \ s_{kj} (t)\right)\nonumber \\
               &-& \sqrt{\rho_b} \sum_{j=1}^{L} \pmb{\widetilde{g}}_{lkl}^T \left(\sum_{k'=1}^{K} \sqrt{\frac{\eta_{jk'}}{M\gamma_{jk'}}} \ \pmb{\hat g}^*_{jk'j} \ s_{kj} (t)\right) 
               + \sqrt{\rho_o}\sum_{j=1}^{L}\pmb{g}_{lkj} \Pi_{\frac{1}{\pmb{\hat G}_j}} \pmb{U}_j \pmb{q}_{j}
               + w_{lk}(t)      
\end{eqnarray}
The first term in (\ref{eqn:completeot}) is the required signal and all other terms constitute to the interference and noise. The second term represents uncertainty in channel gain (UCG), third term for the intra-cell interference, fourth term is the inter-cell interference, the fifth term stems from channel estimation errors, fifth term is the transmission to the O-terminals and the last term is the thermal noise. The signal, interference and the noise are mutually uncorrelated.

Our next aim is to find the SINR by finding the signal power and variances of the noise term. Once SINR is found the rate of transmission to the B-terminal can be calculated.

The signal power(SP) can be calculated from the first term as
\begin{eqnarray}
	SP %&=& \frac{\rho_b\beta_{lp}^{l}\eta_{lp}}{M\gamma_{lp}^{l}}\pmb{E}[||\pmb{\hat  g_{lp}^{l}}||^2]\pmb{E}[(s_{lp}^{l})^Hs_{lp}^{l}] \nonumber \\
    %        &=& \frac{\rho_b\beta_{lp}^{l}\eta_{lp}}{M\gamma_{lp}^{l}}(\pmb{E}[||\pmb{\hat g_{lp}^{l}}||^2])^2 \nonumber \\
            \triangleq \rho_b\eta_{lk}M\gamma_{lkl}
\end{eqnarray}
The variance of UCG is given as
\begin{eqnarray}\label{eq:noiseF1}
	V_1 %&=& \frac{\rho_b \beta_{lkl} \eta_{lk}}{M\gamma_{lk}}   \pmb{E}[|s_{lk}\left(||\pmb{\hat g}_{lkl}||^2- \pmb{E}[||\pmb{\hat  g}_{lkl}||^2] \right)|^2] \nonumber \\
	    &\triangleq& \frac{\rho_b \beta_{lkl} \eta_{lk}}{M\gamma_{lkl}}   \left(\pmb{E}[||\pmb{\hat g}_{lkl}||^4] - (\pmb{E}[||\pmb{\hat g}_{lkl}||^2])^2 \right) \nonumber  \\
	    &=& \rho_b\beta_{lkl}\eta_{lk}\gamma_{lk}
\end{eqnarray}
(In (\ref{eq:noiseF1}), since $\pmb{\hat g}_{lkl} \sim CN(0,\gamma_{lkl})$, the value of $\pmb{E}[||\pmb{\hat g}_{lkl}||^4]$ is evaluated to be $(M)(M+1)\gamma_{lk}^2$).

The intra-cell interference variance is
\begin{eqnarray}\label{eq:noiseF2}
	V_2 &\triangleq& \frac{\rho_b\eta_{lk}}{M\gamma_{lk}} \pmb{E}\left[\Bigm| \pmb{\hat g}_{lkl}  \left(\sum_{\substack{k'=1 \\ k' \neq k}}^{K} \sqrt{\frac{\eta_{jk'}}{M\gamma_{lk'}}} \ \pmb{\hat g}^*_{lk'l} \ s_{lk'l} (t)\right)\Bigm|^2\right] \nonumber \\
	    &=&  \rho_b\gamma_{lk}\sum\limits_{\substack{k'=1\\ k'\neq k}}^{K}\eta_{lk'}
\end{eqnarray}
The variance of inter-cell interference is given as
\begin{eqnarray}\label{eq:noiseF3}
	V_3 &\triangleq& \rho_b \pmb{E}\left[\Bigm|\sum_{\substack{j=1 \\ j \neq l}}^{L} \pmb{\hat g}_{lkl}^T \left(\sum_{k'=1}^{K} \sqrt{\frac{\eta_{jk'}}{M\gamma_{jk'}}} \ \pmb{\hat g}^*_{jk'j} \ s_{kj} (t)\right)\Bigm|^2\right] \nonumber \\
       &=& M\rho_b\gamma_{lk}\sum_{\substack{j=1 \\ j \neq l}}^{L}\sum_{k'=1}^{K}\eta_{jk'}
\end{eqnarray}
The variance of channel estimation error is
\begin{eqnarray}\label{eqn:noiseF4}
	V_4 &\triangleq& \rho_b \pmb{E} \left[ \Bigm |\sum_{j=1}^{L} \pmb{\widetilde{g}}_{lkl}^T \left(\sum_{k'=1}^{K} \sqrt{\frac{\eta_{jk'}}{M\gamma_{jk'}}} \ \pmb{\hat g}^*_{jk'j} \ s_{kj} (t)\right)\Bigm|^2 \right] \nonumber \\
        &=& \rho_b(\beta_{lkl}-\gamma_{lk})\sum\limits_{k'=1}^{K}\eta_{lk'}\gamma_{jk'}
\end{eqnarray}
The variance of the transmission to O-terminals is given by
\begin{eqnarray}\label{eqn:noiseF4_1}
	V_5 &\triangleq& \rho_o\sum_{j=1}^{L}\pmb{E}\left[|\pmb{g}_{lkj}^T\Pi^{\perp}_{{\widehat{\pmb{G}}_j}}\pmb{U}_j \pmb{q}_{j}|^2\right]
\end{eqnarray}

The random variables $\pmb{g}_{lkj}$ and $\pmb{q}_j$ are statistically independent conditioned on $\pmb{U}_j$ and therefore, the expectation in (\ref{eqn:noiseF4_1}) can be computed as a nested expectation, by first taking the expectation conditioned on $\pmb{U}_j$ and then computing the expectation over $\pmb{U}_j$. Thus the variance is derived as
\begin{eqnarray}\label{eqn:noiseF4}
	V_5 &=& 
\end{eqnarray}
The variance of the AWGN noise term is calculated as $1$

%Thus the SINR of the system can be written as 
%\begin{eqnarray}\label{eqn:sinrexp}
%	\pmb{SINR}_{lk}^{MR} &=& \frac{SP_1}{V_1+V_2+V_3+V_4+V_5}
%\end{eqnarray}
%and the rate is 
%\begin{eqnarray}\label{eqn:rateexp}
%	R_{lk}^{MR} &=& log_2(1+\pmb{SINR}_{lk}^{MR})
%\end{eqnarray}
Thus the value of the rate for the $k^{th}$ terminal of the $l^{th}$ cell can be written as
%\begin{eqnarray}\label{eqn:rateot}
%	R_k^{MR} &=& log_2\left[1+ \dfrac{M\rho_b\gamma_{lk}\eta_{lk}}{\rho_b\gamma_{lk}\eta_{lk} + \rho_b\gamma_{lk}\sum\limits_{\substack{k'=1 \\ k' \neq K}}^{K}\eta_{lk'}
%	+ \rho_b\gamma_{lk}\sum\limits_{\substack{j=1 \\ j \neq l}}^{L} \sum\limits_{k' = 1}^{K}\eta_{jk'}
%	+ \rho_b(\beta_{lkl}-\gamma_{lkl})\sum\limits_{j=1}^{L} \sum\limits_{k' = 1}^{K}\eta_{jk'}
%	+V_6
%	+1}\right]
%\end{eqnarray}

\begin{eqnarray}\label{eqn:rateot}
	R_k^{MR} &=& log_2\left[1+ \dfrac{M\rho_b\gamma_{lk}\eta_{lk}}{\rho_b\beta_{lkl}\sum\limits_{j=1}^{L} \sum\limits_{k' = 1}^{K}\eta_{jk'}
				+V_6
				+1}\right]
\end{eqnarray}

\section{Power Allocation Startergy for B-terminals}
  The max-min fairness power allocation scheme shall be followed as described in Chapter 5 and 6.2.5 in \cite{bib:MassiveMimoBook}. The basic principle is to make the SINR for all the terminals in a particular cell same. In the multicell system individual cells shall be considered independently under this stratergy. 

	The stratergy can be explained as follows. Considering the denomintor in (\ref{eqn:rateot}) except the second term all other terms depends directly on the total number of terminals in any particular cell. Thus as long as the number of terminals in a cell is constant these terms an be considered as constant. The second term is the noise from the contaminating cells and depends on the cell under consideration. This is the coherent interference. As the first step in power allocation, we assume that the contamination from other cells is zero, which implies that the denominator is a constant, and design a value for $\eta_{lp}=\hat \eta_{lp}$. Using this value of $\hat \eta_{lp}$ the actual value of SINR (with the coherent interference term) ($\pmb{\widetilde{SINR}}_{lp}^{l}$) is calculated. Dividing this SINR by $\hat \eta_{lp}$ is equivalent to finding the SINR for the terminal with maximum $\eta_{lp}$, i.e 1. Thus an optimum value for $\eta_{lk}$ shall be found by making use of the criteria that, in a particualr cell 
\begin{eqnarray}\label{eqn:etacriteria}
	\sum\limits_{k'=1}^{K}\eta_{lk'}=1
\end{eqnarray}
	For simplicity let us represent the neumerator in (\ref{eqn:rateot}) as $a_{lp}^{l}\eta_{lp} $, first term in denominator as $b_{lk'}^{l}$, second term as $c_{jk'}^{l}$ and third term as $d_{o}^{j}$. Thus (\ref{eqn:rateot}) becomes
\begin{eqnarray}
	R_k^{MR} = log_2\left[ 1+ \frac{a_{lp}^{l}\eta_{lp}}{b_{lk'}^{l}+c_{jk'}^{l}+d_{o}^{j}+L}\right]
\end{eqnarray}
 As described in the stratergy we assume that $c_{jk'}^{l}$ is zero and design for $\hat \eta_{lp}^{l}$ and applying (\ref{eqn:etacriteria}) we get
\begin{eqnarray}
	\hat \eta_{lp}^{l} &=& \frac{L+b_{lk'}^{l}+d_{o}^{j}}{a_{lp}^{l}\sum\limits_{k'=1}^{K}\frac{L+b_{lk'}^{l}+d_{o}^{j}}{a_{lk'}^{l}}}
\end{eqnarray}
Let 
\begin{eqnarray}
	\pmb{SINR}_{lp} &=& \left(\frac{\pmb{\widetilde{SINR}}_{lp}}{\hat \eta_{lp}}\right) \eta_{lp}\nonumber \\
                  &=& \hat f_{lp}\eta_{lp}
\end{eqnarray}
Thus using the condition that all the SINRs of the terminals of a cell shall be equal and (\ref{eqn:etacriteria}), the value of $\eta_{lp}$ is
\begin{eqnarray}
	\eta_{lp} = \frac{\frac{1}{\hat f_{lp}}}{\sum\limits_{k'=1}^{K}\frac{1}{\hat f_{lp}}}
\end{eqnarray}
Substituting the values, the value of $\hat f_{lp}$ is
\begin{eqnarray}
	\hat f_{lp} &=& \dfrac{M\rho_b\beta_{lp}^{l}\gamma_{lp}^{l}}
                            {\rho_b(\beta_{lp}^{l})^2\sum\limits_{k' = 1}^{K}\eta_{lk'}
                            +\rho_b\beta_{lp}^{l} \sum\limits_{\substack{j=1 \\ j \neq l}}^{L} \sum\limits_{k' = 1}^{K}\beta^{j}_{lp}\eta_{jk'}
		  	    +\frac{\rho_o}{M'} \sum\limits_{j=1}^{L}\beta_{lp}^{j}
						   +L}
\end{eqnarray}
and the value of $\eta_{lp}$ is
\begin{eqnarray}
	\eta_{lp} &=&  \dfrac{\rho_b(\beta_{lp}^{l})^2\sum\limits_{k' = 1}^{K}\eta_{lk'}
                            +\rho_b\beta_{lp}^{l} \sum\limits_{\substack{j=1 \\ j \neq l}}^{L} \sum\limits_{k' = 1}^{K}\beta^{j}_{lp}\eta_{jk'}
		  	    +\frac{\rho_o}{M'} \sum\limits_{j=1}^{L}\beta_{lp}^{j}+L}
			     {\beta_{lp}^{l}\mathlarger{\mathlarger{\sum}}\limits_{k''=1}^{k''=K}\dfrac{
                            \rho_b(\beta_{lk''}^{l})^2\sum\limits_{k' = 1}^{K}\eta_{lk'}
                            +\rho_b\beta_{lk''}^{l} \sum\limits_{\substack{j=1 \\ j \neq l}}^{L} \sum\limits_{k' = 1}^{K}\beta^{j}_{lk''}\eta_{jk'}
		  	    +\frac{\rho_o}{M'} \sum\limits_{j=1}^{L}\beta_{lk''}^{j}
						   +L}{\beta_{lk''}^{l}}}
\end{eqnarray}

\section{Performance of the O-terminals}
	The O-terminals receive the signal which is a combination of data to the B-terminals, the data for the O-terminal and the noise. We assume the O-terminal experiences independent Rayleigh fading such that $\pmb{h_{l}} \sim CN(0,\pmb{C_h}$) where $\pmb{C_h} = \beta_{ol}I$. The channel from the base station to the terminals shall be estimated first. Using this estimated value, with JBB, the received signal at the O-terminal shall be evaluated to find the signal power. noise variances and the achievable rate.

\subsection{Pilot Phase}
	In the pilot phase of the O-terminals the base station sends the pilots to the O-terminals. Here we assume that in the pilot phase there is no transmission to the B-terminals.
The signal received by the O-terminal in the $l^{th}$ cell is given by
\begin{eqnarray}
	y_{ok}(t)=\sqrt{\rho_o} \sum_{j=1}^{L} \left[\pmb{h}^H_{ej} \ \pmb{q}_{j}(t) + w_{oj}(t) \right]
\end{eqnarray}
where
\begin{eqnarray}
	\pmb{h}_{ej} &\triangleq& \pmb{U}_j^H\Pi_{\frac{1}{\pmb{\hat G}_j}}\pmb{h}_j \nonumber \\
    \pmb{E}[\pmb{h}_{ej}\pmb{h}^H_{ej}|\pmb{\hat G}_j] &=& \beta_{ol}\pmb{I} \nonumber \\
			                     &=& \pmb{E}[\pmb{h}_{ej}\pmb{h}^H_{ej}] \nonumber \\
                                           &\triangleq& \pmb{C}_{\pmb{h}_{ej}} \nonumber
\end{eqnarray}
At the $k^{th}$ cell, O-terminal correlates $y_{ok}(t)$ with the pilot and the resulting signal is
\begin{eqnarray}
	\pmb{y}_{ok}^{p} &\triangleq& \sum_{t=1}^{t=\tau_p^o} y_{ok}^*(t)\pmb{q}^{p}_{j}(t) \nonumber \\
\end{eqnarray}
where $\pmb{q}_j^p(t)$ are the pilot signals sent at time $t$. The energy spent per sample is same during the pilot and payload phases. Hence 
\begin{eqnarray}
    \sum\limits_{t=1}^{\tau_p^o}\pmb{q}_{pj}(t)\pmb{q}_{pj}(t)=\frac{\tau_p^o}{M'}\pmb{I}
\end{eqnarray}
\begin{eqnarray}
\implies \pmb{y}_{ok} &=& \sqrt{\rho_o}\frac{\tau_p^o}{M'_j}\sum_{j=1}^{L}\pmb{h}_{ej}+\pmb{n}_{pk}(t)
\end{eqnarray}
where $\pmb{n_{pk}}(t) = \sum\limits_{j=1}^{L} \sum\limits_{t=1}^{\tau_p^o} w_{oj}(t) \pmb{q_{pj}}(t)$. The variance of the AWGN is calculated as
\begin{eqnarray*}
	\pmb{E}[\pmb{n_{pj}(n_{pk}})^H] &=& \frac{L\tau_p^o}{M'}I 
\end{eqnarray*}
Following the same procedure as in section (\ref{sec:btchesti}) the channel from the $l^{th}$ cell base station to the O-terminal can be derived as 
\begin{eqnarray}\label{eqn:otchesti}
	\pmb{\hat h_{el}} &=& \frac{M'\sqrt{\rho_o\beta_{ol}}}{M'L+\rho_o\tau_p^o\sum_{j=1}^{L}\beta_{oj}}\pmb{y_{pl}}
\end{eqnarray}
Similarly the variance of the estimate and error variance are
\begin{eqnarray}
	\pmb{C_{\hat h_{ej}}} &\triangleq& \pmb{E}[\pmb{\hat h_{ej}}(\pmb{\hat h_{ej}})^H] = \frac{\rho_o\tau_p^o\beta^2_{ol}}{\rho_o\tau_p^o\sum_{j=1}^{L}\beta_{oj}+M'L}\pmb{I} \\
	\pmb{C_{\widetilde{h}_{ej}}} &=& \beta_{ol} - \pmb{C_{\hat h_{ej}}} 
\end{eqnarray}

\subsection{Payload Phase}

The received signal at the O-terminal in the $l^{th}$ cell is given by
\begin{eqnarray}\label{eqn:otyt}
 y_{ol}(t) = \sqrt{\rho_o} \sum_{j=1}^{L} (\pmb{\hat h_{ej}})^H \pmb{q_j(t)}
            + \sqrt{\rho_b'} \sum_{j=1}^{L} (\pmb{h_j})^H \left(\sum_{k'=1}^{K} \pmb{v_{jk'}} s_{jk'}(t) + \sum_{j=1}^{L}W_{oj}(t)  \right)
\end{eqnarray} 
The equation (\ref{eqn:otyt}) can be written in terms of the channel estimate and error values.
\begin{eqnarray}\label{eqn:otcomplete}
  y_{ol}(t)= \sqrt{\rho_o} (\pmb{\hat h_{el}})^H \pmb{q_l}(t) 
	    - \sqrt{\rho_o}(\pmb{\widetilde{h}_{el}})^H\pmb{q_l}(t)
	    +\sqrt{\rho_o} \sum_{\substack{j=1 \\ j \neq l}}^{L} (\pmb{h_{ej}})^H\pmb{q_{j}}(t) \nonumber \\
            + \sqrt{\rho_b'} \sum_{j=1}^{L} (\pmb{h_j})^H \left(\sum_{k'=1}^{K} \pmb{v_{jk'}} s_{jk'}(t) 
	    + \sum_{j=1}^{L}W_{oj}(t)  \right)            
\end{eqnarray}
The first term in (\ref{eqn:otcomplete}) is the required signal and other terms constitute to the noise. To find the SINR, the signal power along with the noise variances shall be calculated. Please note that the signal is uncorrelated with noise and the noise themselves are uncorrelated.
The variances of all terms are computed conditioned on $\pmb{\hat h_{ej}}$. The signal power is given as
\begin{eqnarray}
	E_1 = \rho_o\pmb{E}[|(\pmb{\hat h_{el}})^H \pmb{q_l}(t)|^2 | \pmb{\hat h_{el}}] &=& \rho_o\pmb{E}[||\pmb{\hat h_e}||^2 | \pmb{\hat h_{el}}] \nonumber \\
                                                                                           &=& \frac{\rho_o}{M'} ||\pmb{\hat h_{el}}||^2
\end{eqnarray}
The noise variances for the other terms in (\ref{eqn:otcomplete}) are as follows
\begin{eqnarray}\label{eqn:otv1}
	V_1 &=& \pmb{E}[|\sqrt{\rho_o}(\pmb{\widetilde{h}_{el}})^H \pmb{q_l}(t)|^2|\pmb{\hat h_{el}}] \nonumber \\
            &=& \frac{\rho_o}{M'} \pmb{E}[||\pmb{\widetilde{h}_{el}}||^2 |\pmb{\hat h_{el}}] \nonumber \\
            &=& \frac{\rho_o}{M'}\left(\beta_{ol}-\frac{\rho_o\tau_p^o\beta^2_{ol}}{\rho_o\tau_p^o\sum_{j=1}^{L}\beta_{oj}+M'L}\right)          
\end{eqnarray}
\begin{eqnarray}\label{eqn:otv2}
	V_2 &=& \rho_o \sum_{\substack{j=1 \\ j \neq l}}^{L} \pmb{E}[|(\pmb{h_{ej}})^H\pmb{q_j}(t)|^2] \nonumber \\
            &=& \frac{\rho_o}{M'} \sum_{\substack{j=1 \\ j\neq l}}^{L} \beta_{oj}
\end{eqnarray}
\begin{eqnarray}
	V_3 &=& \rho_b' \sum_{j=1}^{L} \pmb{E}\left[\left|\sum_{k'=1}^{K} (\pmb{h_j})^H \pmb{v_{jk'}}s_{jk'}(t) \right|^2 | \pmb{\hat h_{el}} \right] \nonumber \\
            &=& \rho_b' \sum_{j=1}^{L} \pmb{E}\left[\sum_{k'=1}^{K} (\pmb{v_{jk'}})^H \pmb{h_j}  (\pmb{h_j})^H \pmb{v_{jk'}}| \pmb{\hat h_{el}} \right] \nonumber \\
            &=& \rho_b' \sum_{j=1}^{L} \pmb{E}\left[\pmb{E}\left[\sum_{k'=1}^{K} (\pmb{v_{jk'}})^H \pmb{h_j}  (\pmb{h_j})^H \pmb{v_{jk'}}| \pmb{\hat h_{el}},  \pmb{\hat G_{j}} \right] | \pmb{\hat h_{el}}\right] \nonumber \\
            &=& \rho_b' \sum_{j=1}^{L} \pmb{E}\left[\ (\pmb{v_{jk'}})^H \pmb{E}\left[\sum_{k'=1}^{K} \pmb{h_j}  (\pmb{h_j})^H | \pmb{\hat h_{el}},  \pmb{\hat G_{j}} \right] \pmb{v_{jk'}}| \pmb{\hat h_{el}}\right]
\end{eqnarray}
The inner expectation can simplified using the equality \cite{bib:rmtBook}
\begin{eqnarray}
	\pmb{E}[\pmb{h_j}(\pmb{h_j})^H|\pmb{\hat h_{ej}}, \pmb{\hat G_{j}}] = \pmb{C_{h_j}}-\pmb{E[h_j (\hat h_{ej})^H \hat| G_{j}] - C_{\hat h_{ej}}^{-1}E[(\hat h_{ej})^H h_j| \hat G_{j}}]
\end{eqnarray}
where
\begin{eqnarray}
	\pmb{E[(\hat h_{ej})^H h_j| \hat G_{j}}] = \frac{\beta_{oj}^{2}\rho_o\tau_p^o}{M'L+\rho_o\tau_p^o\sum\limits_{j=1}^{L}\beta_{oj}}\pmb{U_j}
\end{eqnarray}
Thus
\begin{eqnarray}\label{eqn:otv3}
	V_3 &=& \rho_b' \sum_{j=1}^{L} \left[ \beta_{oj} \pmb{E}[\sum_{k'=1}^{K}||\pmb{v_{jk'}}||^2]
					- \frac{\beta_{ol}^{2}\rho_o\tau_p^o\pmb{U_j}}{M'L+\rho_o\tau_p^o\sum\limits_{j=1}^{L}\beta_{oj}} \pmb{E}\left[\sum_{k'=1}^{K} (\pmb{v_{jk'}})^H \pmb{U_j}  (\pmb{U_j})^H \pmb{v_{jk'}}\right]\right] \nonumber \\
	    &=& \rho_b'\sum_{j=1}^{L}\beta_{oj}
\end{eqnarray}

For the AWGN term, the noise variance is 
\begin{eqnarray}
	V_4 &=& \sum\limits_{j=1}^{L}\pmb{E}[|w_{oj}(t))|^2] \nonumber \\
            &=& L
\end{eqnarray}
The rate for the O-terminal is
\begin{eqnarray}\label{eqn:rateotfull}
	R_o &\triangleq& \pmb{E_{\hat h_{el}}}\left[log_2\left(1+\frac{\frac{\rho_o||\pmb{\hat h_{el}||^2}}{M'}}{V_1+V_2+V_3+V_4}\right) \right] \nonumber \\
            &=&  \pmb{E_{\hat h_{el}}}\left[log_2\left(1+\frac{\frac{\rho_o||\pmb{\hat h_{el}||^2}}{M'}}
								 {\frac{\rho_o}{M'}(\beta_{ol}-\frac{\rho_o\tau_p^o\beta^2_{ol}}{\rho_o\tau_p^o\sum_{j=1}^{L}\beta_{oj}+M'L})        
								  + \frac{\rho_o}{M'} \sum_{\substack{j=1 \\ j\neq l}}^{L} \beta_{oj} 
								  +\rho_b'\sum_{j=1}^{L}\beta_{oj}
								  +L}\right) \right]	
\end{eqnarray}
To obtain a simple closed-form bound of the expectation in (\ref{eqn:rateotfull}) we use the fact that if $\Psi$ is an $M'$-vector with independent $CN(0,1)$ elements, then for any $\alpha>0$
\begin{eqnarray}\label{eqn:bound}
	\pmb{E}[log_2(1+\alpha||\Psi||^2)] &\geq& log_2\left(1+\frac{\alpha}{E[\frac{1}{||\Psi||^2}]}\right) \nonumber \\
                                              &=&  log_2(1+(M'-1)\alpha)
\end{eqnarray}
The result in (\ref{eqn:bound}) is obtained from \cite{bib:dtsysBook}. Using the value of the variance of the estimate, bound for the rate can be formulated as
\begin{eqnarray}
	R_o &\geq& \left[log_2\left(1+\frac{\frac{\rho_o(M'-1)}{M'}\frac{\rho_o\tau_p^o\beta^2_{ol}}{\rho_o\tau_p^o\sum_{j=1}^{L}\beta_{oj}+M'L}}
								 {\frac{\rho_o}{M'}(\beta_{ol}-\frac{\rho_o\tau_p^o\beta^2_{ol}}{\rho_o\tau_p^o\sum_{j=1}^{L}\beta_{oj}+M'L})        
								  + \frac{\rho_o}{M'} \sum_{\substack{j=1 \\ j\neq l}}^{L} \beta_{oj} 
								  +\rho_b'\sum_{j=1}^{L}\beta_{oj}
								  +L}\right) \right]	
\end{eqnarray}

%\begin{appendices}
%\section{Notations} \label{app:notation}
%\begin{flushleft}
%\begin{tabular}{ll}
%	$\pmb{h_{j}}$                  & Channel between the O-terminal in the $l^{th}$ cell and the base station of $j^{th}$ cell  \\ \\ 
%	$\pmb{g_{jk}^{l}}$             & Channel between the $k^{th}$ B-terminal of the $j^{th}$ cell and base station of the $l^{th}$ cell  \\ \\
%	$\pmb{\hat g_{jk}^{l}}$        & Estimate of the channel  $\pmb{g_{jk}^{l}}$ \\ \\
% 	$\pmb{\widetilde{g}_{jk}^{l}}$ & Error in channel estimate of $\pmb{g_{jk}^{l}}$ \\ \\
%	$\pmb{\eta_{jk}}$              & Power control coefficient of the $k^{th}$ terminal in $j^{th}$ cell \\ \\
%        $\pmb{\beta_{jk}^{l}}$         & Fading between $k^{th}$ terminal in $j^{th}$ cell and the base station in $l^{th}$ cell. \\ \\
%        $\pmb{\beta_{oj}}$             & Fading between the O-terminals of the $l^{th}$ cell and base station in $j^{th}$ cell. \\ \\
%        $\pmb{\gamma_{jk}^{l}}$        & Mean square channel estimate between $k^{th}$ terminal in $j^{th}$ cell and the base station in $l^{th}$ cell. \\ \\
%        $\pmb{\phi_{jk}^{l}}$          & Pilot symbols send from $k^{th}$ B-terminal in $j^{th}$ cell to the base station in $l^{th}$ cell. \\ \\
%        $\pmb{v_{jk}^{j}}$             & Beamforming vector from base station in $j^{th}$ cell to the $k^{th}$ B-terminal in $j^{th}$ cell \\ \\
%        $\pmb{s_{jk}^{j}}$             & Information symbols from base station in $j^{th}$ cell to the $k^{th}$ B-terminal in $j^{th}$ cell \\ \\
%	$\pmb{q_{oj}}$                 & Information symbols send to the O-terminal from the base station of $j^{th}$ cell.\\ \\
%	$\pmb{q_{pj}}$                 & Pilot symbols send to the O-terminal from the base station of $j^{th}$ cell.\\ \\
%	$W_{oj}$                          & Noise for the O-terminals in $l^{th}$ cell when transmitting fron $j^{th}$ terminal. \\ \\
%        $\tau_{p}$                        & Pilot sending duration for B-terminals.\\ \\
%        $\tau_{p}^{o}$                    & Pilot sending duration for O-terminals.\\ \\
%        $N_{LK}^{l}$                      & Noise from all the $K$ terminals from all the $L$ cells.\\ \\
%	$\pmb{\hat G_j}$               & Estimate of channels of B-terminals in $j^{th}$ cell
%\end{tabular}
%\end{flushleft}
%\end{appendices}

\bibliography{jbb}{}
\bibliographystyle{plain}
\enddocument
